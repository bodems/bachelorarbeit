\documentclass{scrartcl}
\usepackage[utf8]{inputenc}
\usepackage[ngermanb]{babel}
\usepackage{graphicx}
\usepackage{amssymb}
\usepackage{amsmath}
\usepackage{wrapfig}
\usepackage{gensymb}
\usepackage{cite}
\usepackage{float}
\usepackage{url}
\usepackage{lscape}
\usepackage[onehalfspacing]{setspace}
\usepackage{helvet}
\renewcommand{\familydefault}{\sfdefault}

\title{Simulation dynamischer Vorgänge im elektrischen Netz mit PSS Sincal und PSS Netomac}
\author{Felix Annen}



\begin{document}

\begin{titlepage}



\maketitle
\thispagestyle{empty}
\newpage
	\subsection*{Eidesstattliche Erklärung}
	\glqq Ich Versichere, dass ich diese Studienarbeit selbstständig und nur unter Verwendung der angegebenen Quellen und Hilfsmittel angefertigt und die benutzen Quellen als solche kenntlich gemacht habe. Die Arbeit hat in gleicher oder ähnlicher Form noch keiner Prüfungsbehörde vorgelegen\grqq \\ \\ \\
	Bielefeld, den \today


\end{titlepage}


	\pagenumbering{Roman}
	\setcounter{page}{1}
	\tableofcontents
	\newpage
	\listoffigures
	\listoftables
	\section*{Abkürzungsverzeichnis}
	
	\newpage
	\setcounter{page}{1}
	\pagenumbering{arabic}
\begin{onehalfspace}

\section{Einleitung}
\section{aktueller Stand}
\section{Grundlagen}
\subsection{Fehlerarten in elektrischen Netzen}
Als Fehler bezeichnet man eine Abweichung vom ungestörten Betriebszustand eines Netzes (Qualle?). Der bekannteste Fehler ist der Kurzschluss, bei dem mindestens zwei Potentialpunkte niederohmig miteinander verbunden sind und ein wesentlich größerer Strom als der Nennstrom fließt (KS bei PV?). Diese hohen Ströme verusachen einen Spannungseinbruch an der Fehlerstelle und, bei entsprechender Zeit, Beschädigungen an Betriebsmitteln. In Energieversorgungsnetzen entstehen Fehler meistens durch Bäume, bei Kabeln meistens durch Bauarbeiten (Quelle?). Gespeißt werden diese Kurzschlussströme vor allem durch Synchrongeneratoren, in kleinerem Umfang auch durch Asynchrongeneratoren. Synchronmotoren können im ersten Moment wie Synchrongeneratoren betrachtet werden, das selbe gilt für Asynchrongeneratoren und Asynchronmaschinen. Kategorisiert werden können Kurzschlüsse einmal durch die Anzahl und die Verbindungen der am Kurzschluss beteiligten Pole und durch die elektrische Entfernung zu Generatoren. Bei ersteren Unterscheidet man zwischen:

\begin{itemize}
\item Dreipoliger Kurzschluss
\item Zweipoliger Kurzschluss ohne Erdberührung
\item Zweipoliger Kurzschluss mit Erdberührung
\item Einpoliger Erdkurzschluss
\item Doppelerdschluss
\end{itemize}

Eine Besonderheit in Netzen der Mittel- und Hochspannung bis 110kV ist, dass diese oftmals gelöscht betrieben werden. Das heißt, dass einige Transformatorsternpunkte mit einer Erdschlusslöschspule geerdet sind, die parallel zur Leitungskapazität wirkt. Der Fehlerstrom eines einpoligen Fehlers wird durch die Kompensation wesentlich geringer, sodass die Chance steigt, dass sich der dazugehörige Lichtbogen von selbst löscht. Der Erdkurzschluss wird zu einem Erdschluss.

Das zweite Unterscheidungsmerkmal ist, ob der Fehler \glqq generatornah \grqq{} oder \glqq generatorfern\grqq{} auftritt. Dazu werden die Teilkurzschlussströme betrachtet, die die Generatoren beitragen. Übersteigt mindestens einer dieser Teilkurzschlussströme, die durch einen Generatoren fließen, den doppelt Wert des Nennstromes des Generators, so gilt der Kurzschluss als generatornah. Das Kriterium \glqq generatornah oder generatorfern\grqq{} ist also ein Merkmal für die Größe der Netzimpedanz zwischen Generatoren und Fehlerstelle.

\subsection{Kurzschlussströme}
Bei der Berechnung der Kurzschlussströme unterscheidet man zwischen verschiedenen Kurzschlussgrößen. Gemeinsam haben sie alle, dass sie aus dem Anfangskurzschlusswechselstrom $I_k''$ abgeleitet werden. Der Anfangskurzschlusswechselstrom ist laut VDE 0102 der \glqq Effektivwert zum Zeitpunkt des Kurzschlussbeginns\grqq.

\subsubsection{Anfangskurzschlusswechselstrom}

\subsubsection{Transienter Kurzschlusswechselstrom}

\subsubsection{Stoßkurzschlussstrom}
Der Stoßkurzschlussstrom $I_s$ ist der Strom mit der höchsten Amplitude, der bei einem Kurzschluss auftritt. Er berechnet sich aus dem Spitzenwert des Anfangskurzschlusswechselstromes multipliziert mit einem Stoßfaktor $\kappa$. Dieser liegt zwischen 1,0 und 2,0 und hängt vom R/X-Verhältnis des Netzes ab (der Kurzschluss... S. 246). Die Höhe des Stoßkurzschlussstromes hängt aber auch vom 

\subsubsection{Dauerkurzschlussstrom}
Der Dauerkurzschlussstrom $I_k$ beschreibt den Effektivwert des Stromes, der nach dem Abklingen aller Ausgleichsvorgänge fließt. Beim generatornahen Kurzschluss ist dieser größer als der Anfangskurzschlusswechselstrom $I_k''$. Er berechnet sich wie folgt: \\

$I_k = \frac{U_n}{\sqrt{3} \cdot X_d}$ \\

 Beim generatorfernen Kurzschluss ist $I_k = I_k''$. Die Ausgleichsvorgänge haben nur einen geringen Einfluss und werden vernachlässigt.

\subsubsection{Ausschaltwechselstrom}
Beim Ausschaltwechselstrom wird zwischen dem unsymmetrischen und symmetrischen Ausschaltwechselstrom unterschieden. Dieser beschreibt den Strom, der beim Ausschaltvorgang über einen Schalter fließt. Da in der Regel die Ausgleichsvorgänge noch nicht abgeschlossen sind, fließen in die Berechnung von $I_b$ die Zeit und die  subtransiente und transiente Zeitkonstante mit ein: \\

$I_b = (I_k'' - I_k') \cdot e ^{-t/T_d''} +  (I_k' - I_k) \cdot e ^{-t/T_d'} + I_k$ \\

Ia == alt Ib == neu

Ip Bedeutung?\\
Ik \\
Ik' \\
Ik'' \\
Ith \\

\subsection{Dynamischer Verlauf des Kurzschlussstroms}
Der Verlauf eines Kurzschlussstromes kann aufgeteilt werden in einen Daueranteil, einen transienten Anteil und einen subtransienten Anteil. Diese ergeben überlagert den in Abbildung \ref{kss-verlauf} dargestellten Verlauf.

	\begin{figure}[H]
	\centering
	\includegraphics[scale=1]{img/kurzschlussstromverlauf-nah.jpg}
	\caption{Verlauf des generatornahen Kurzschlussstroms (Quelle: el. Kraftw. und Netze)}
	\label{kss-verlauf}
	\end{figure}

Beim Eintritt des Kurzschlusses wird als erstes der subtransiente Anteil sichtbar. Er klingt bereits nach ein einigen Halbwellen exponentiell mit der Zeitkonstanten $T_g''$ ab und geht in den transienten Kurzschlussstrom über, welcher wesentlich langsamer abklingt. Überlagert wird der Kurzschlussstrom noch durch einen Gleichstromanteil. Dieser kommt dadurch zustande, dass durch die überwiegend induktive Kurzschlussreaktanz der Strom nicht \glqq springen\grqq{} kann. Aus diesem Grund wird ein Ausgleichsstrom $i_g$ erzwungen, welcher mit der Gleichstromzeitkonstanten $T_g$ abklingt. Die Höhe des Gleichstromgliedes ist abhängig vom Einschaltwinkel, also vom Einschaltzeitpunkt des Kurzschlusses, $\psi$ und ist bei $\psi = 0$ am größten.
\\- Gleichstromglied

\subsection{Einpolige Fehler}
Einpolige Fehler mit Erdberührung sind die am häufigsten auftretenden Fehler in elektrischen Netzen (Quelle?). Sie entstehen beispielsweise durch herabfallende Äste oder durchhängende Vogelnester. Üblicherweise werden Netze bis 110kV kompensiert betrieben. Dabei wird der kapazitive Anteil des Stromes an der Fehlerstelle durch eine Petersenspule kompensiert. Kleinere Mittelspannungsnetze, beispielsweise in Industrieanlagen, werden auch mit isoliertem Sternpunkt betrieben.


\subsection{Dynamische Größen der Synchronmaschine}
Bei der dynamischen Betrachtung von Kurzschlüssen kommt der Synchronmaschine eine Besonderheit zu. Da in konventionellen Kraftwerken ausschließlich Synchrongeneratoren zum Einsatz kommen, bestimmen diese maßgeblich den Verlauf des Kurzschlussstromes. Des weiteren ist die wirksame Reaktanz der Synchronmaschine während der Dauer eines Kurzschlusses nicht konstant, das heißt, die gesamte Kurzschlussimpedanz ändert sich ständig.

\subsubsection{Aufbau der Synchronmaschine}
Eine Synchronmaschine besteht, wie eine Asynchronmaschine, aus einem Ständer und einem Läufer. Der Aufbau des Ständers ist analog zur Asynchronmaschine. Bei einer Polpaarzahl von Eins sind in einem räumlichen Abstand von 120\degree um den Läufer drei Wicklungen angeordnet. Bei höheren Polpaarzahlen entsprechend sechs, neun, usw. Beim Aufbau des Läufers wird zwischen Schenkelpol- und Vollpolläufern unterschieden. Kleine und langsam laufende Generatoren (Laufwasserkraftwerke) bis ca. 300MVA besitzen meist einen Schenkelpolläufer. In großen schnell drehenden Kraftwerksgeneratoren sind Vollpolläufer verbaut, da Schenkelpolläufer nicht die enormen Fliehkräfte aushalten können. 
\\ (el. netze u. kraftwerke: S. 126 ff)

\subsubsection{d-q-Achse}
Die Reaktanz einer Synchronmaschine setzt sich aus einem Anteil der Längsachse (d-Achse) und einem Anteil der Querachse (q-Achse) zusammen. Bei der Schenkelpolmaschine ist die magnetische Leitfähigkeit der d-Achse größer als die der q-Achse, was zur Folge hat, dass Xd größer ist als Xq. Die ideale Vollpolmaschine ist mathematisch gesehen ein Sonderfall der Schenkelpolmaschine, bei der Xd = Xq ist. Für eine einfache Kurzschlussstromberechnung sind die Reaktanzen der d-Achse ausreichend. Will man jedoch dynamische Vorgänge simulieren, so werden die Reaktanzen der q-Achse ebenfalls benötigt. \\

	\begin{figure}[H]
	\centering
	\includegraphics[scale=0.85]{img/schenkelpol.jpg}
	\includegraphics[scale=0.85]{img/vollpol.jpg}
	\caption{Querschnitt Schenkelpol- und Vollpolläufer (Quelle: el. Kraftw. und Netze)}
	\end{figure}

	\begin{figure}[H]
	\centering

	\caption{test}
	\end{figure}
%Ersetzt man nach der Zweiachsentheorie die drei Ständerwicklungen der Synchronmaschine durch zwei Ersatzwicklungen, so sind diese 90\degree elektrisch zueinander verschoben. Diese Komponenten auch als d-Achse (Längsachse) und q-Achse (Querachse) bezeichnet. Betrachtet man unsymmetrische Vorgänge, kommt noch eine 0-Komponente hinzu. Die Umrechnung der Drehstromkomponenten in dq0-Komponenten geschieht mit der Park-Transformation. Warum nur xd und kein xq bei KS-Berechnung?


\subsubsection{Subtransiente Reaktanz}
Um ersten Moment eines Kurzschlusses ist die subtransiente Reaktanz $X_d''$ wirksam. Mit ihr wird der Anfangskurzschlusswechselstrom berechnet. Ihr zugeordnet ist die subtransiente Zeitkonstante $T_d''$, welche das abklingen des Anfangskurzschlusswechselstromes beschreibt. Der Abklingvorgang wird mit einer e-Funktion beschrieben. Der gesamte subtransiente Vorgang dauert nur einige Halbwellen, danach geht der Kurzschluss in den transienten Vorgang über

\subsubsection{Synchrone Reaktanz}
Die synchronen Reaktanz $X_d$ gibt die wirksame Reaktanz während des stationären Betriebs an. Sie ist auch nach dem Abschluss aller Ausgleichvorgänge wirksam, daher kann mit ihr der Dauerkurzschlussstrom ermittelt werden. Die Reaktanz der q-Achse wird entsprechend mit $X_q$ bezeichnet.

\subsubsection{Transiente Reaktanz}
Während des transienten Vorgangs, der wesentlich langsamer Abklingt als der subtransiente Vorgang, ist die transiente Reaktanz $X_d'$ wirksam.


d-q \\
el. netze u. kraftwerke: S. 127 \\
xd'' \\
xd' \\
xd \\
Td'' \\
Td' \\
Tg \\

\subsection{Berechnung}

\subsection{Stabilität}

\section{PSS Sincal}

\section{Simulation mit Netomac}
\subsection{Export von Sincal}
\subsection{Programmstruktur}
\subsection{Eingabedaten}
\subsection{Festlegen der Störkriterien}
\subsection{CTL-Datei}
\subsection{Grafische Darstellung}

\section{Prüfgerät Kocos Artes}

\section{Überarbeitung Laborunterlagen}
\section{Zusammenfassung}

\newpage
\pagenumbering{Roman}
\setcounter{page}{1}
\section{Literaturverzeichnis}
\bibliography{bibtest1}
\bibliographystyle{unsrtdin}

\end{onehalfspace}

\end{document}

